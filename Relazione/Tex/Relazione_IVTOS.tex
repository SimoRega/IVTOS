\documentclass[a4paper,12pt]{report}

\usepackage{alltt, fancyvrb, url}
\usepackage{graphicx}
\usepackage[utf8]{inputenc}
\usepackage{hyperref}

% Questo commentalo se vuoi scrivere in inglese.
\usepackage[italian]{babel}

\usepackage[italian]{cleveref}

\title{Relazione per il progetto di\\``Basi di Dati''}

\author{Linda Fabbri,\\Federico Raffoni,\\Simone Rega}
\date{\today}


\begin{document}

\maketitle

\tableofcontents

\chapter{Introduzione}

Il progetto consiste nella realizzazione di un sistema database che funga da supporto alla creazione di Tornei Internazionali di Videogiochi.
Il database ha l'obiettivo principale di immagazzinare le informazioni relative a: videogiochi, giocatori e partite. 
L'applicazione permetterà la creazione di vari tornei in tutto il mondo consultando statistiche dei giocatori nei vari videogiochi e cercando il luogo migliore in cui ospitarli, ovvero con strutture adeguatamente attrezzate e tenendo conto dell'audience e sponsor locali.


\chapter{Analisi dei Requisiti}
La seguente descrizione riporta in linguaggio naturale i requisiti per il nostro sistema informativo, per poi poterne estrarre i principali concetti fondamentali:
\section{Requisiti in linguaggio naturale}
"Jeff Kaplan, prima di lasciare le redini del videogioco Overwatch, ha deciso di commissionare un sistema informativo di supporto per la gestione di tornei internazionali di cui finanzierà i premi.
Si vuole tenere traccia dei giocatori iscritti, memorizzandone nome, cognome, nickname, codice fiscale, stato in cui risiede, mail e statistiche di gioco (per statistiche si intendono il livello/rank e le ore di gioco).
Un giocatore può participare a uno o più tornei come membro di una squadra, per quanto riguarda i tornei si memorizzano: stato, città e arena in cui si svolge, numero di squadre totali e videogioco per cui si disputa il torneo in questione.
Di ogni Videogioco si vuole tener traccia del Nome, della data di creazione ,della sua azienda produttrice, della tipologia di gioco e del numero di componenti di ogni squadra.
In ogni Arena possono assistere alle Partite un numero massimo di spettatori, i quali per poter assistere dovranno pagare un biglietto nominativo; saranno inoltre presenti vari Sponsor e Speaker che commenteranno il torneo in tempo reale.
\section{Estrazione dei concetti fondamentali}


\chapter{Progettazione Concettuale}

\section{Anteprima Schema Scheletro}
\section{Anteprima sviluppo delle "Persone"}
\section{Anteprima sviluppo dei "Videogiochi"}
\section{Anteprima sviluppo delle "Partite"}
\section{Anteprima sviluppo dei "Tornei"}
\section{Schema Generale}


\chapter{Progettazione Logica}
\section{Stima del volume dei dati}
\section{Descrizione delle operazioni principali e stima della loro frequenza}
\section{Schemi di navigazione e tabelle degli accessi}
\section{Raffinamento dello schema}
\subsection{Eliminazione Gerarchie}
\subsection{Eliminazione attributi composti}
\subsection{Scelta delle Chiavi}
\section{Analisi delle ridondanze}
\section{Traduzione di entità e associazioni in relazioni}
\section{Schema relazionale finale}


\chapter{Progettazione Fisica}
\section{Traduzione in SQL}


\chapter{Progettazione dell'Applicazione}
\section{Descrizione della scelta del linguaggio e del DBMS}
\section{Descrizione dell'architettura}
\section{Interfaccia Utente}
\subsection{Amministratore Torneo}
\subsection{Giocatore}
\end{document}
